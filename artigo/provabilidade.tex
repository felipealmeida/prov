%% -*- TeX-engine: xetex; mode: LaTeX; eval: (TeX-PDF-mode); coding: utf-8-unix; TeX-master: t; ispell-local-dictionary: "pt_BR"; -*-

\documentclass[12pt]{article}

\usepackage[brazil]{babel}
\usepackage{sbc-template}

\usepackage{graphicx,url}

\usepackage{fontspec}
\usepackage{color}
\usepackage{listings}

%\usepackage[brazil]{babel}   
%\usepackage[latin1]{inputenc}  

%\sloppy

\lstset{basicstyle=\small,language=C++,stringstyle=\ttfamily,showstringspaces=false,keywordstyle=\color{blue}}

\title{Provabilidade em C++}

\author{Felipe Magno de Almeida}

\address{Expertise Solutions}

\begin{document} 

\maketitle
     
\begin{resumo} 

  Provabilidade

\end{resumo}

\section{Introdução}

\section{Estado}

É possível pensar, formalmente, em estado de um programa de computador
como um conjunto de pares. Aonde o primeiro elemento do par é
identifica a variável de nosso programa, e.g. por identificador, e o
segundo elemento do par o valor dessa variável naquele estado.



\section{Calculo de predicado}



\end{document}
